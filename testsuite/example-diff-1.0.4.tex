\documentclass[12pt,a4paper]{article}
%DIF LATEXDIFF DIFFERENCE FILE
%DIF DEL example-old.tex   Thu Jun 12 00:01:26 2014
%DIF ADD example-new.tex   Thu Jun 12 00:01:26 2014

\setlength{\topmargin}{-0.2in}
\setlength{\textheight}{9.5in}
\setlength{\oddsidemargin}{0.0in}

%DIF 7c7
%DIF < \setlength{\textwidth}{6.5in}
%DIF -------
\setlength{\textwidth}{6in} %DIF > 
%DIF -------

\title{latexdiff Example - \DIFdelbegin \DIFdel{Draft }\DIFdelend \DIFaddbegin \DIFadd{Revised }\DIFaddend version}
\author{F Tilmann}
% Note how in the preamble visual markup is never used (even %DIF > 
% if some preamble might eventually end up as visible text.) %DIF > 
%DIF PREAMBLE EXTENSION ADDED BY LATEXDIFF
%DIF UNDERLINE PREAMBLE %DIF PREAMBLE
\RequirePackage[normalem]{ulem} %DIF PREAMBLE
\RequirePackage{color}\definecolor{RED}{rgb}{1,0,0}\definecolor{BLUE}{rgb}{0,0,1} %DIF PREAMBLE
\providecommand{\DIFadd}[1]{{\protect\color{blue}\uwave{#1}}} %DIF PREAMBLE
\providecommand{\DIFdel}[1]{{\protect\color{red}\sout{#1}}}                      %DIF PREAMBLE
%DIF SAFE PREAMBLE %DIF PREAMBLE
\providecommand{\DIFaddbegin}{} %DIF PREAMBLE
\providecommand{\DIFaddend}{} %DIF PREAMBLE
\providecommand{\DIFdelbegin}{} %DIF PREAMBLE
\providecommand{\DIFdelend}{} %DIF PREAMBLE
%DIF FLOATSAFE PREAMBLE %DIF PREAMBLE
\providecommand{\DIFaddFL}[1]{\DIFadd{#1}} %DIF PREAMBLE
\providecommand{\DIFdelFL}[1]{\DIFdel{#1}} %DIF PREAMBLE
\providecommand{\DIFaddbeginFL}{} %DIF PREAMBLE
\providecommand{\DIFaddendFL}{} %DIF PREAMBLE
\providecommand{\DIFdelbeginFL}{} %DIF PREAMBLE
\providecommand{\DIFdelendFL}{} %DIF PREAMBLE
%DIF END PREAMBLE EXTENSION ADDED BY LATEXDIFF

\begin{document}
\maketitle

\section*{Introduction}

This is an extremely simple document that showcases some of \DIFaddbegin \DIFadd{the }\DIFaddend latexdiff features.
Type
\begin{verbatim}
latexdiff -t UNDERLINE example-draft.tex example-rev.tex > example-diff.tex
\end{verbatim}
to create the difference file.  You can inspect this file directly. Then run either 
\begin{verbatim}
pdflatex example-diff.tex
xpdf example-diff.pdf
\end{verbatim}
or
\begin{verbatim}
latex example-diff.tex
dvips -o example-diff.ps example-diff.dvi
gv example-diff.ps
\end{verbatim}
to display the markup.

\section*{\DIFdelbegin \DIFdel{Another }\DIFdelend \DIFaddbegin \DIFadd{Yet another }\DIFaddend section title}

 \DIFdelbegin \DIFdel{A paragraph with a line only in the draft document.  }\DIFdelend More things could be said were it not for the constraints of time and space.

\DIFaddbegin \DIFadd{A paragraph with a line only in the revised document.  }\DIFaddend More things could be 
said were it not for the constraints of time and space.

And here is a \DIFdelbegin \DIFdel{tipo}\DIFdelend \DIFaddbegin \DIFadd{typo}\DIFaddend . 

Here is a table:

\begin{tabular}{ll}
Name & Description \\
\hline
Gandalf & \DIFdelbegin \DIFdel{Grey }\DIFdelend \DIFaddbegin \DIFadd{White }\DIFaddend \\
Saruman & \DIFdelbegin \DIFdel{White
}\DIFdelend \DIFaddbegin \DIFadd{Evil
}\DIFaddend \end{tabular}

And \DIFdelbegin \DIFdel{sometimes a whole paragraph gets completely rewritten. In this
case latexdiff marks up the whole paragraph even if some words in it
are identical}\DIFdelend \DIFaddbegin \DIFadd{now for something completely different, with not a paragraph in sight}\DIFaddend .
No change, 
no markup!
\end{document}


