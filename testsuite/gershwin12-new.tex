\documentclass[jgr]{agu2001}
%\documentclass[draft,jgr]{agu2001}
%\usepackage{times}
\bibliographystyle{agu}
\usepackage{mylatex}
% uncomment the next two lines to include graphics
\usepackage{graphicx}
%\usepackage{mygraphicx}

\newlength{\tw}
\setlength{\tw}{\textwidth}
\newcommand{\tm}{\tablenotemark}
%\renewcommand{\includegraphics}[2][]{\fbox{Include: {\sf #1} {\tt #2 }}}
% comment unless pictures are in draft mode
\setkeys{Gin}{draft=false}

\authorrunninghead{TILMANN ET AL.}
\titlerunninghead{SEISMICITY OF MID-ATLANTIC RIDGE, 5\dg S}

%\authoraddr{F. J. Tilmann,
%Bullard Laboratories, Department of Earth Sciences,
%Madingley Road, Cambridge CB3 0EZ, United Kingdom 
%(tilmann@esc.cam.ac.uk)}

%\authoraddr{Ernst Flueh, Lars Planert, Tim Reston, Wilhelm Weinrebe,
%GEOMAR, Wischofstr. 1-3, 24148 Kiel, Germany
%(\mbox{eflueh@geomar.de}, \mbox{lplanert@geomar.de}, \mbox{treston@geomar.de},\linebreak \mbox{wweinreb@geomar.de})}

%\received{}
%\revised{}
%\accepted{}
%\cpright{AGU}{2003}

\begin{document}

\title{Micro-earthquake seismicity of the Mid-Atlantic ridge at
 5\dg S, a view of tectonic extension}
%\author{Frederik Tilmann}
%\affil{GEOMAR, Kiel; Bullard Laboratories, Department of Earth Sciences,
%University of Cambridge}
%\author{Ernst Flueh, Lars Planert, Tim Reston, Wilhelm Weinrebe}
%\affil{GEOMAR, Kiel}
\date{\today}

\begin{abstract}
Based on an  ocean-bottom micro-earthquake survey of the
Mid-Atlantic ridge just south of the 5\dg S transform fault/fracture
zone, we find seismic activity to be concentrated within the western
half of the median valley.  The median valley seismic zone is bounded
in along-axis direction by the transform faults to the north and the
north, and the tip of the axial volcanic ridge to the south.  A few
scattered events occurred within the inside corner high, on the
transform fault, and in the western side wall close to the segment
center. Earthquakes reach a maximum depth of 8~km below the median
valley floor and appear to be predominantly in the mantle although a
few crustal earthquakes also occurred.  We infer the median valley
seismic activity to arise from slip on two parallel low-angle normal
faults which dip from the inside corner toward the spreading axis.
\end{abstract}

\begin{article}
%\setlength{\baselineskip}{9pt}

\section{Introduction}

Magmatism and mechanical extension are 
 processes contributing to  seafloor spreading.  Whereas magmatic injection is dominant for
fast-spreading ridges mechanical extension is likely to be important
in slow-spreading ridges  \citep{mutter92}.  Evidence for the strong
role of mechanical extension at slow-spreading ridges can be seen in the morphology of the
spreading axis (well developed median valleys with at least one
bounding side wall with strong relief), the intermittent nature of
magmatic activity (no continuous eruption centers are generally 
discernible, and seismically detectable magma chambers, common at
fast-spreading ridges, are usually absent), the higher rate and larger
depth extent of seismic activity (both in local surveys
\citep[e.g.,][]{toomey88} and globally \citep{huang88,rundquist02}),
and the strong segmentation by transform faults and non-transform
ridge discontinuities (NTD). A strong asymmetry along both types of
segment boundary is frequently observed, with the inside corner next
to the active transform or NTD being
characterized  by high topography, large inferred fault spacings, and a positive gravity anomaly
indicative of thinned crust, and the outside corner next to the
inactive fracture zone being associated with more subdued topography,
small inferred faults spacings, 
and a negative gravity anomaly \citep{shaw93,escartin95}. This asymmetry,
combined with the observation of corrugations parallel to the
spreading direction on many inside corner highs (ICH) and the
recovery of gabbro and serpentinite samples from their surface, has
led to a model where ICHs are interpreted to be the unroofed footwalls
of  deeply penetrating detachment faults and  extension
is thus largely accommodated by simple shear
\citep{tucholke94}. Shallow listric successor faults might then
develop in the ICH to accommodate bending as result of progressive footwall rotation.  In this model, the outside
corner is characterised by high angle, small offset normal faulting
between rotated fault blocks. In one flavour of this model, a magmatic
phase follows once the offset on the detachment has reached a critical
value, which must be sufficiently large to exhume mantle peridotite in
order to explain the presence of serpentinite dredged from ICHs. The crust 
created in the magmatic phase is then cut by a new detachment fault,
and the cycle is repeated.  Alternatively,
magmatic production is contemporaneous with mechanical extension and part
of the newly created crust is continuously exhumed by the
detachment. In this model, the ICH is primarily gabbroic at all
stages, and the serpentinite samples are interpreted as fault gouge
smeared along the slip surface of the mantle penetrating detachment \citep{reston02}.
Consistent with both models, quasi-periodically spaced topographic and
gravity highs can be
found away from the Mid-Atlantic ridge  (MAR) near fracture zones on the
inside corner side with wavelengths of 10-30~km along the flowline, i.e. in
ridge-perpendicular direction, giving a time-scale of 0.6--2~My for initiation of new detachment faults.  The
association of these highs with detachment faults was supported by the
observation of a sequence of low-angle (30\dg) faults in a multi-channel seismic
line along a fossil inside corner flowline, where the faults were
associated with basement topography reminiscent of
present-day ICHs \citep{ranero99}. 
Towards the segment center, where fault spacing is usually small and
the across-axis profile more symmetric \citep{shaw93}, the
detachment is thought to die out because of thermal structure or as a
geometric requirement.  The geometric requirement only applies to
segments where the transform or NTD steps in the same direction at
both ends; an inside corner is thus paired with an outside corner, and
the segment center needs to mediate between the different structures.
Being further away from the cooling influence of the transform or NTD,
the segment centers are likely to be warmer, and the lithosphere is
accordingly weaker.  The median valley within segment centers is often
elevated by 1000~m or more compared to the median valley near the
segment boundaries, presumably because of increased
crustal thickness near the center,
which in turn is related to the  higher temperatures just discussed.

Seismicity patterns can provide a direct image of the tectonic
processes currently operating, and give indirect information about the
temperature through the depth of the brittle-ductile transition.
Here we report on the micro-earthquake activity of the Mid-Atlantic
ridge segment just south of 5\dg S based on a brief survey carried out
as part of R/V Meteor
cruise M47/2 in 2000, also drawing on several wide-angle profiles shot
through the array [{\it Planert et al.}, manuscript in preparation].
The segment is bounded to the north by the 70 km long left-laterally
stepping 5\dg S transform fault and fracture zone
(Fig.~\ref{fig:glob-seis}).  The off-axis bathymetry is marked by a
number of topographic features trending obliquely to the direction of
both ridge and transform faults. These linear features are particularly visible on the South
American plate, where they trend predominantly NE-SW; they are probably
related to regular rearrangements of the spreading system or migrating
NTDs between the Ascension and 5\dg S transform faults. 
 The study area is unusual in that the ICH elevation is almost matched
by that of the OC massif,
albeit the latter is of much smaller lateral extent
(Fig.~\ref{fig:seismag-map}).  This peculiarity and a number of other
morphological features have led \citet{reston02} to propose that the
outside corner massif is really part of a fossil ICH which was split
by a ridge jump to the west at $\sim$0.75~Ma.  A similarly-shaped,
though less elevated massif further to the east hints at the
possibility that several instances of ridge jumping might have
occurred in the past. \remark{TIM, IN THE EPSL ARTICLE YOU STATE
SOMETHING DIFFERENT, IE. THAT OC LOOKS 'NORMAL' AFTER INITIAL
HIGH. HOWEVER IT SEEMS TO ME THAT THE NEXT HIGH GOING EAST IS MORPHOLOGICALLY
QUITE SIMILAR TO THE OC, AND I SEEM TO REMEMBER THAT IN CONVERSATIONS
YOU WERE NOT OPPOSED TO THE IDEA OF MULTIPLE RIDGE JUMPS, IN FACT IT
MIGHT HAVE BEEN YOUR IDEA}  In spite of
this peculiarity we will argue that in fact the seismicity patters are
broadly consistent with previous surveys along the MAR and can be
understood in terms of a variation of the basic
\citet{tucholke94} model:  the seismicity suggests
that at least two parallel low angle faults are active at the same time
rather than extension being taken up by one detachment only.

\section{Earthquake location}

\subsection{Data}

A network of altogether 16 free fall ocean bottom stations (consisting
of 13 ocean bottom hydro\-phones--OBH), and 3 ocean bottom seismometers/hydro\-phones--OBS) was deployed
on May 3, 2000 on the inside corner high and the median valley in the
study area.  The array was continuously recording at 100 Hz sampling frequency, while other
geophysical and geological investigations (bathymetric mapping,
dredging, refraction seismology) were carried out in the study area.
The total recording period was 10 days, however, 3 instruments
returned no usable data because of equipment problems, and a number of
stations recorded only for a few days, mainly because they were
recovered before the end of the experiment in order to be used in the
contemporaneous refraction experiments.  Because of the rough
topography and lack of sedimentary cover at the ridge the seismometers
did not couple well to the seafloor, resulting in ringing and delayed
signal onsets, such that few usable velocity seismograms were
obtained.  The hydrophones were partly differential pressure gauges
(DPG) and partly piezo-electric hydrophones.
%henceforth simply referred to as ``hydrophones''.  
They generally produced clear signal onsets for
P waves and even recorded waves converted from S to P at the seafloor
for some events and stations
(Figure~\ref{fig:data-example}). Accordingly, most subsequent analysis
was based on the hydrophone and DPG recordings.  An
exception to this rule is station obs12, which recorded reasonable Z
component seismograms, but no usable hydrophone data.  

 In spite of the short recording time a large number of events was
recorded. Basic processing involved the following steps:
\begin{enumerate}
\item Relocation of stations using the water wave arrival time of
airgun shots.
\item 
Correction of the timing of the records assuming  linear drift
of the data logger clock between synchronization with GPS time at the
beginning and end of the experiment.  
\item Bandpass filter with a 5-20 Hz passband.  The
filtering is crucial for the DPG records, for which otherwise
microseismic noise completely obscures the signal. We checked with
hydrophone records that signal distortion and any delay introduced by
the filter are not severe.  
\item Generation of  a preliminary list of events
with a trigger algorithm that  detects nearly
co-incident changes in the amplitude at several stations and can
detect and remove man-made airgun shots, which otherwise lead to a
large number of spurious triggers.  
\item Manually inspect all trigger events and pick arrivals, assigning
a weight to each pick. Remove
events, which are unclear, presumably not earthquakes, or cannot be
picked on at least three stations. 
\item Obtain a preliminary location of each event by linearised
inversion and using a 1-D velocity model derived from the refraction
data [{\it Planert et al.}, manuscript in preparation].
\end{enumerate}
Altogether 148 events were pickable on at least 3 instruments. Of
these, 77 had picks on at least 5 stations and an azimuthal gap of
less than 300\dg, and are referred to as ``restricted event set''. The
remaining 71 events are termed ``marginal events''. (An azimuthal gap
of 300\dg is not normally considered to be sufficient for a
well-constrained location; however, extensive testing of the
robustness of these locations with respect to random errors and the
assumed velocity models showed that they are well enough located to
provide meaningful tectonic information, and should therefore be included.)
28 events in the
restricted data set
 had at least one S pick, and 52 events had reasonably well
constrained depth (see below for further details). 

\subsection{Method}

We re-located all events in the restricted event set at the same time as determining a minimum
1D-model and station corrections \citep[using VELEST,][]{kissling94}.  Since the event set is rather small, and most events
fall into the 7--11 km depth band, the 
minimum 1D model is poorly resolved, and only the average sub-Moho
velocity, which is almost 8~km/s,
is constrained by the inversion.
%  (All depths are reported relative
% to sea-level; in order to get depth below the median valley seafloor
% it is necessary to subtract 4--5 km.)
%  Consequently, the only quantity constrained by
%the inversion is the average sub-Moho velocity, which is almost
%8~km/s. 
 Otherwise, the model is strongly influenced by the starting 
model.  We considered two extreme starting models, one appropriate for
the median valley, where---as we will see---most earthquakes are located, and
one appropriate for the ICH, where most stations are located (see
Figure~\ref{fig:velmod} for the velocity models). We also
relocated the events keeping the model fixed, i.e. by a simple
joint-hypocenter determination .  All inversions achieve a satisfactory fit of
the data. We chose the minimum 1D model resulting from an inversion with the
ICH starting model as the preferred model because it yielded the
lowest residual RMS.  In most instances the systematic differences in
inferred location between the different velocity models is small, such
that the patterns reported in the following are not affected by the
choice of velocity model (see appendix).
However, the apparent depth of shallow median valley earthquakes is
increased by about 1~km in the median valley model compared to their
location in the preferred model.  

The picking error was estimated {\it a posteriori} by the method of
\citet{wilcock91}, modified to allow for different pick weights, to be 0.04~s
for the highest quality picks, and 0.08~s for the lowest quality picks.
% that were still used in the location procedure.
The location uncertainty due to
picking errors is evaluated using a Monte-Carlo method, and again all
results reported in the following are robust with respect to likely
mislocations (see appendix for details).  
Subsequently, we re-located all marginal events with the station
terms and, where applicable, the velocity model derived from the
restricted data set. 

We further re-located the events using the
double-difference method \citep{waldhauser00}.  Whereas the results
were broadly consistent with the joint hypocenter or minimum 1D model
locations, they did not concentrate seismicity in narrower bands, and
relative relocation vectors between the double-difference method and
the conventional techniques showed no systematic pattern.  A failure
of the double-difference method to improve location accuracy in this
experiment would not be surprising
because of the relatively small number of picks available for most
events. In the absence of further information we thus preferred the
locations obtained by the use of absolute travel times.

\subsection{Results}

The distribution of events is shown in
Figure~\ref{fig:seismag-map}. All events with both upper
and lower 68\% confidence bounds on depth of less than 2~km are
included in the cross-sections and maps with grayshade-coded earthquake
depth in Figures~\ref{fig:croslon} and \ref{fig:croslat}.
Traditionally, it is assumed that the availability of $S$ picks is
critical for earthquake depths to be constrained, whereas we only have
$S$ picks available for a subset of those events.  However, for many
events both direct and indirect arrivals have been recorded at
different stations.  Because these arrivals are associated with rays leaving
the focus in upward and downward direction, respectively, the
partial derivatives of their arrivals times with respect to depth have
opposite signs, and depth can thus be reasonably well constrained even
in the absence of $S$ waves.
It has to be kept in mind that the seismic activity patterns described
in the following strictly apply only to the duration of the
experiment. We will consider later how representative they might be of
seismicity in the longer term.  Only the events in the median valley
are sufficiently numerous and well-constrained to warrant a detailed
discussion.  We will thus defer discussion of these events to a later
section but briefly discuss the other groups here.
\begin{description}
\item[Median Valley, along-axis deep]  The vast majority of events during the
experiment occurred within the median valley, referred to henceforward
as the median valley seismic zone (MVSZ).  Earthquakes are
concentrated within a 5--8~km wide zone bounded to the west by a
fault, labelled F1 in Figure~\ref{fig:seismag-map}, which was
previously identified from the bathymetric data \citep{reston02}.
This fault has to be considered the side wall of the median
valley, even though it is only associated with rather minor topography
(0.5--1~km) compared with the imposing scarp (more than 2~km relief)
presented by the split ICH massif.  To the north the seismic zone is
bounded by the transform fault; an apparent reduction of activity
within 5~km of the transform is not robust when considering the
location uncertainties of the earthquakes and the station
distribution. To the south the seismic zone terminates rather sharply near the
northern tip of the axial volcanic ridge at 5\dg 14'S.  The latitude
of earthquakes near the southern tip is rather well constrained.  We
further tested the influence of the station distribution on this limit
by re-locating all events but removing a number of stations a time.
Even when leaving out stations obs12, obh11, and obh10 (the
southernmost row) the southern limit of the MVSZ
does not change.  The eastern limit of the seismically active zone is
somewhat less well defined.  North of 5\dg 10' S, most earthquakes are
in the western half of the valley with a few events near the eastern
flank.  South of  5\dg 10'S, earthquakes occur under the whole
width of the median valley. Earthquakes depths lie between 5 and 12~km
below sea-level, or equivalently 1-8~km below the median valley floor.  The pattern
of activity is virtually identical for the restricted and marginal
event set, which gives reassurance that no significant bias was
introduced by the selection criteria.
  The two marginal events
underneath the outside corner have uncertainties large enough that
they might have occurred in the outer reaches of the median valley. Hence, we
do not consider the outside corner as a separate seismically
active zone.
% Earthquakes in the mantle 
% association with faults

\item[Inside Corner High]  A small number of events occurred
underneath the ICH.  The  4 events with depth control have depths
between 7 and 10~km below sea level, or equivalently 4--8~km
below the seafloor, putting them in the mantle and lower
crust (see Figure~\ref{fig:croslat}, top and middle). 
Therefore, they cannot be associated with the formation of successor
faults in the \citet{tucholke94} model, which are shallow.
A micro-earthquake survey of a  seismically much more active ICH
\citep[at 29\dg N,][]{wolfe95} has also placed ICH events in the
mantle, and has determined one composite focal mechanism showing
normal faulting with a $\sim$45\dg dip.  \citet{wolfe95} interpreted the ICH
seismicity to result from tectonic extension within a
diffuse zone underneath the ICH with no single dominant detachment
fault.  Analogously, we propose that the few events in the ICH
accommodate tensile stresses within the ICH, although we have no focal solution to
ascertain the validity of this statement.  In any case, the low level
of seismic activity, if indeed representative, argues that extension
within the ICH is minor compared to the processes operating underneath the
median valley.

\item[Western flank, close to segment center]
A cluster near the southernmost station obh16
comprises 6 events in the restricted set and 3 marginal events.
The 4 earthquakes with well-constrained depths lie between  6 and
10~km below sea level on an eastward
dipping plane (about 30\dg dip), but the number of events is too small to identify this
plane with a fault plane.
%The fact that the events are distributed in depth but are confined
%laterally to a small area is suggestive of high-angle faulting,
%although again no focal mechanism is available to support or refute
%this conjecture.  Such a pattern would be consistent with the
%classical model for segment centers of
%slow-spreading ridges \citep{mutter92}
% where new crust is continuously generated
%within the median valley and is accompanied by
%high-angle faulting within the side walls.

\item[Transform Fault]  A number of events are likely to have
originated on the transform fault.  These include the two events in the
restricted set just north of the ICH Eastern scarp with depths of
7 and 7~km, and the three
large marginal events near 5\dg 02'S, 11\dg 57'W, the uncertainties of
which are all consistent with a origin on the transform. In fact, it
is remarkable how close to the transform they locate in spite of  the
considerable distance from the network.  The only event that locates  north of the transform is
particularly poorly constrained, such that it could have also
originated on the transform.  
%The events on the transform probably
%accommodate transverse motion between the African and South American
%plate according to the classic model of transform fault seismicity \citep{sykes67}.
\end{description}

\section{Focal mechanisms}

First motion polarities were determined on the unfiltered records, if
possible.  However, it was necessary to apply a 1-20 Hz bandpass
filter to the DPG data for
all but the largest events in order to make the signal visible above
the micro-seismic noise. As the application of this particular bandpass did not
change first motion polarities of those arrivals which could be picked on
the unfiltered records, we are confident that our results are not
biased by the use of the bandpass.  
No individual event provided enough measurements to constrain the
focal solution with any degree of confidence. Nevertheless, some
systematic trends were discernible:  allowable pressure
axes are either close to vertical or their horizontal component is approximately aligned with the
median valley direction, and allowable tension axes were in general within
30\dg or so of the horizontal direction.  Assuming double-couple
mechanisms, composite focal solutions
were then determined for groups of closely spaced events with similar
waveforms.   Independent and stable solutions could be obtained for
only two groups of events (Table~\ref{tbl:focmec} and Figure~\ref{fig:seismag-map})
because most of the events occurred at the edge of the array and
because take-off angles are strongly
model-dependent at close distances.

The solution for the group of 5 events within the center of the
MVSZ exhibits eastward dipping low-angle normal
faulting (or westward-dipping high angle faulting)
with a strike parallel to the strike of the median valley bounding
fault F1 (340\dg) or perpendicular to the spreading direction (347\dg in
the Nuvel-1 model \citep{demets90})---uncertainties are too large to
discriminate between both possibilities.
The same solution is obtained for both end member velocity models,
i.e. the median valley and the ICH velocity models.

The solution for the group of 6 events near the southern end of
the MVSZ shows some model dependence, and two alternative, albeit similar
solutions are presented.  The first is obtained for the median valley
velocity model and shows extremely low angle normal faulting ($<10\dg$
eastward dip), or extremely high-angle faulting verging on a dip-slip
mechanism, striking 10\dg, i.e. rotated clockwise from the strike of
the solution for the first event group but almost parallel to the southward continuation
of bounding fault F1, which exhibits a kink near 5\dg 16'S,  a few km south of the
event group.  The second solution is obtained for the ICH model and
shows oblique normal-faulting, again with a strike closer to
perpendicular to the spreading direction (350\dg) but still rotated
clockwise with respect to the  solution to the first event.

\section{Earthquake Magnitude}

The absolute calibration of our hydrophones was not well determined,
and we were also concerned about possible site effects due to
the rough topography.  Therefore, we did not determine seismic moments
directly, but instead measured peak-to-peak amplitudes within 15~s of
the first arrival for each station on the 5--20~Hz bandpass filtered,
but not instrument-corrected
records, i.e. the unit of the amplitude measurements is counts.
  We then assumed these measurements
can be described by \citep[modified from][]{hutton87}
\begin{equation}
M_i=\log A_{ij} + B \log(d_{ij}) + C d_{ij} - S_{j}
\end{equation}
where $M_i$ is the magnitude of the $i$th earthquake, $A_{ij}$ the
amplitude of earthquake $i$ at station $j$, $d_{ij}$ the distance
between the hypocenter and the station, $B$ is a parameter related to
geometric spreading ($B$=1 corresponds to body wave spreading in a
homogeneous medium), $C$ is a parameter related to attenuation, and
$S_{j}$ is the station parameter ($S_{j}$ corresponds to the logarithm
of the product of the station gain, including any site effects, and a
constant that relates physical amplitude measurements, i.e. pressure
or displacement, to the magnitude scale).

The equation above can be written as 
\begin{equation}
\log A_{ij}= M_i - B \log(d_{ij}) - C d_{ij} + S_{j}
\end{equation}
Treating $M_i$, $B$, $C$, and $S_j$ as unknowns and calculating
$d_{ij}$ for each station event-pair, we inverted the resulting linear
system of equations for the restricted event set, and subsequently
used the values of $B$ ($1.08 \pm 0.13$), $C$ ($0.011\pm0.03$) and the
% with newer event set (inv.pro109kmvel instead of default), the
% following values are obtained:
% 'b'    [1.2804]    [0.1230]   'c'    [0.0074]    [0.0035]
% (1 standard deviation errors)
$S_{j}$'s to determine magnitudes for the full set.   
Although no significance should be attached to the values of $B$ and
$C$, the fact that they are reasonable gives confidence in the
applicability of the equations above.

Because there is a trade-off between the average
value of the $M_i$'s and the average value of the $S_j$'s, the absolute
magnitudes are unconstrained by this approach and need to be fixed.
In order to get a least some idea about absolute magnitudes we made
use of two independent
approaches. First, we extrapolated the long term globally registered earthquake activity on this
part of the MAR to determine how many events of a certain magnitude
would be expected for the duration of our experiment. We then fixed the
average magnitude such that the number of events recorded matches the
expected number.  Second, we measured corner frequencies on the DPG
traces for all
events in the restricted set, after first correcting for the
frequency-dependent response of the instruments. Assuming a stress drop (here 5~bar) and the
\citet{brune70} source model, we can estimate moment
magnitudes from the corner frequencies. We then fixed the average magnitude of our estimate, such
that magnitudes agree with the moment magnitudes thus estimated.
Both approaches involve a number of poorly determined unknowns and
uncertain assumptions, but agree to within about half a magnitude with
each other.  We thus consider the reported absolute magnitudes
accurate to within 0.5--1 magnitude steps.  Relative magnitudes are
determined much more reliably with 95\% confidence uncertainties
of 0.12--0.24 magnitude steps.  
% with more up-to-date event set get 0.13-0.21.

We obtained a $b$ value of  $1.27\pm0.14$ for the events in the MVSZ (95\%
confidence error of straight line fit, Figure~\ref{fig:magnitude}).  
% newer value for MV events only: b=1.269\pm 0.135
The $b$ value is the slope of
the cumulative magnitude-frequency distribution and is this only
dependent on relative magnitudes. It measures the
relative frequency of large and small earthquakes where a large $b$
values is associated with a large number of small events for a given
number of larger events.
Sometimes, $b$ values are quoted for frequency-magnitude  plots with 
 the logarithm of seismic moment as the dependent variable. In order to convert the magnitude $b$ values into a
 moment $b$ value it is necessary to multiply by a factor of $2/3$,
yielding $b=0.8\pm0.1$.
% b=0.0846\pm0.090  for updated set

\section{Refraction experiment}

Four long intersecting wide-angle profiles (up to 170 km) and three
shorter  long profiles (up to 50 km) were acquired during the cruise.  The long
lines run both parallel and perpendicular to the median valley. 
The first set of lines intersect beneath the ICH, the second set
concentrates on the segment north of the 5\dg S transform fault (Figure.
%extend from the center of one segment across the transform well into
%the next segment.
 The short profiles  run nearly in north-south
direction and make use of the stations distribution of the seismological network
(see Figure~\ref{fig:seismag-map}). For the long lines a total of 54 GEOMAR ocean bottom
seismometers (OBS) and hydrophones (OBH) with an average spacing of
3nm recorded the airgun shots. An array with a total volume of 96l
was used as the seismic source. A shot interval of 60s and a ship
speed of 4kn resulted  in an average  shot spacing of 120m. 
% The four long profiles focused on a number of topographic features
% (median valley, inside corner high,  outside corner, transform fault)
% in order to resolve related velocity variations inside the crust and
% the uppermost mantle. Imaging the internal velocity structure
% contributes to the understanding of the tectonic processes occuring at
% slow spreading ridge transform intersections. 
 
Because the rough topography caused enhanced attenuation of the seismic
energy propagation efficiency varies strongly but in most cases
arrivals can be seen for offsets of more than 40km, sometimes up to 90km. Besides crustal phases and
mantle phases (Pn), only a few Moho reflections (PmP) are visible in
the data. For assessing velocity models we chose a combination of
forward modelling and first-arrival tomographic inversion. For
profiles with sufficient Moho reflections a joint refraction and
reflection travel-time tomography was employed. \remark{REFERENCES:
ZELT?, KORENAGA?} 

%Preliminary results show a velocity structure which differs
%significantly from normal oceanic crustal structure
%\citep[e.g.]{white92}.
 In the median valley the models show an unusually thin crust
of about 4km thickness (near obh04, see Figure~\ref{fig:croslon})
underlain by a normal-to-low
velocity upper mantle ($V_p\sim$7.5~km/s). \remark{I RE-COLLECT THIS
WAS SUPPOSED TO BE A MINIMUM ESTIMATE. IS THIS STILL TRUE?} The Moho is shallowing
towards the transform fault in the north, reaching a depth of about
7km below sealevel at the northern end of the line, corresponding to a
crustal thickness of 3~km. Velocity depth profiles show no strong changes in the velocity
gradient, starting with  sub-seafloor velocities of about 3km/s
increasing to lower crustal velocities of max. 6.9km/s.
 In contrast
velocity depth profiles in the region of the inside corner high as
well as the outside corner show either very high near surface
velocities (>6.0km/s near obs06, see Figure~\ref{fig:croslat}) or very high velocity
gradients resulting in  crustal velocities up to 6.5km/s within the
first 1000m below seafloor. Below, velocities  increase steadily up to
 7.5-7.8km/s at 4.0--6.0 km depth below seafloor. Due to the
absence of clear $P_mP$ reflections, the Moho is not well defined. By
assuming that velocities of $\geq$7.5km/s are indicative for the upper
mantle, models suggest  crustal thicknesses of 4.0-5.0km at the
eastern flank of the inside corner high and 4.5-5.5km at the outside
corner. Further refinements including amplitude modelling is still in
progress and will be presented elsewhere. \remark{CAN I GET A MOHO
DEPTH CONTOUR, OR IS THE MODEL NOT WELL CONSTRAINED ENOUGH}


\section{Discussion}

\subsection{How representative are the results of the long term
  seismicity and comparison with teleseismically recorded earthquakes}

The interpretation of  seismicity patterns derived from brief deployments requires
the implicit or explicit assumption that these patterns are indicative
of long term trends.
%, even though swarm-like activity is known to
%occur frequently for large earthquakes (magnitude 5--6) at mid-oceanic
%ridges \citep{sykes70}.   
Given the very short period our array was
active we have to carefully consider the question of applicability to longer
time frames.  Since even the teleseismic record is unlikely to
span a full seismic cycle, there is no way to answer this question with
 certainty.  Nevertheless, we will put forward a number of
circumstantial arguments to make the case that the assumption of long-term
applicability is at least plausible.   First, we can probably discount the
possibility that the recorded micro-earthquakes are largely
aftershocks of a large earthquake just before the deployment of the
ocean bottom network.  We estimate that an earthquake rupturing the whole
MVSZ would have a magnitude of
6.5.  Even an event rupturing along a length of 10 km would still have
a magnitude of 4.5--5.  Earthquakes of such magnitude are large enough
to be recorded globally, but no event near the study area is listed in
the NEIC catalog \citep{neic}
either immediately preceding the experiment, or shortly afterwards.
The fact that magnitudes estimated from the corner frequency agree at
least in order-of-magnitude with the rate of events expected from
extrapolation of the Gutenberg-Richter curve for events in the global
data set for this part of the MAR also argues against the possibility that
we caught an aftershock or swarm sequence with
seismicity rates strongly exceeding long term averages.  As outlined
in the preceding section, the
absolute magnitude estimate is crude, though,  and  small changes with respect to the
background rate would go unnoticed.
Second, if we discount the effect of events being masked by
co-incident airgun shooting and the change in array geometry towards
the end of the experiment, seismic activity is well distributed in
time, i.e., there is no correlation between the timing of an event and
its location, and no swarm-like activity is noticeable. 
Our third argument relies on the precedence set by a number of
micro-earthquake surveys along the Northern MAR
\citep{toomey88,kong92,wolfe95,barclay01}, lasting between 10 and 43
days. The seismicity obtained in these experiments can now be compared
to the seismicity as determined by an array of autonomous hydrophones
in the North Atlantic \citep{smith03}, which has been operating for
since 2000. Both the spatial pattern
\citep[Fig.~13 of ][]{smith03} and rate of seismic
activity (Table~\ref{tbl:comparison}) show approximate agreement between
micro-earthquake surveys and hydrophone data, if systematic shifts in
the epicentral positions are attributed to topographic steering (the
location by hydrophones relies on T phases, i.e., strictly describes the
point of conversion from a $P$ wave within the solid earth to a
acoustic wave in the SOFAR channel rather than the epicenter).  This broad agreement indicates that
Mid-Atlantic Ridge seismicity recorded over periods of days to weeks is
usually representative of seismicity over a period of a few years.
How this relates to seismicity over several centuries is still
an open question that current data does not allow us to address.

Figure~\ref{fig:glob-seis} gives an overview of the globally recorded
seismicity in the vicinity of the study area. Judging from the
apparent location of the strike-slip events in the south of the map,
which are presumably all associated with the Ascension transform
fault, epicentral mislocation of the Harvard centroid moment tensor (CMT) solutions
\citep{dziewonski81a} can be as large as 50~km. The events 
in the EHB catalog, which were located with short period body waves
\citep{engdahl98}, follow the bathymetric trace of the ridge  transform
faults more closely with formal standard errors of 10--20~km.  Only
one centroid moment tensor solution locates within the study
area. Like the composite focal mechanisms it exhibits (oblique) normal
faulting. However, the dip is $\sim 45\dg$ for both nodal planes,
15--20\dg steeper than the low-angle plane inferred from the local
composite solution. However, the reported centroid depth is 15~km,
which is very likely too deep given the results of this and other OBS
surveys and  maximum centroid depth limit
of 6--7~km for slow spreading ridges that \citet{huang88} inferred  from waveform modeling. 
The dip trades off with depth and is poorly constrained by the CMT
method if the depth is incorrect
\citep{dziewonski81a}\remark{STATEMENT REALLY PERS. COMM. FROM
  ALESSIA MAGGI, BUT SHE THINKS DZIEWONSIKI SAYS SO, TOO. FJT: CHECK} such that
the discrepancy between the CMT catalog earthquake and the composite focal
solutions is probably not significant. Similarly to the results of the
ocean bottom survey, the ridge is far more seismically active than the
transform.  Some earthquakes locate between 5\dg 15' and 5\dg 30'S,
where the median valley  was found to be inactive in the ocean bottom
survey, but location uncertainties are too large to tell whether these
earthquakes were in the median valley or along its flanks. There the
ocean bottom network did record some events, so there is no
inconsistency between the distribution of the teleseismically located
events, and those recorded in the ocean bottom
survey.

\subsection{Tectonic interpretation}

Within the MVSZ, earthquakes predominantly occur on the western half of the median valley but are
fairly uniformly distributed in north-south direction.  Whereas some
crustal earthquakes have occurred, most of the events have hypocenters
beneath the Moho, even when taking into account uncertainties within
their location and inaccuracies in the Moho depth, which was obtained
by wide-angle modelling.  The deepest earthquakes reach a depth of 
12~km, or equivalently 8~km beneath the median valley floor. The ridge-parallel cross section
along the median valley (Figure~\ref{fig:croslon}) suggests a
shallowing of the base of the seismogenic zone both towards the
volcanic ridge at the segment center and towards the segment end.
However, this apparent shallowing might be an artifact of the
selection criteria for events with well-constrained depths: the areas
around obh03 (near the segment end) and around obh09 (near
the southern end of the MVSZ) contain many more
events with poorly constrained depths than the region
near obh04 (near where the deepest earthquakes are observed).  This
pattern is not surprising, as the accuracy of depth determination
depends on good azimuthal coverage and the presence of stations with
direct and refracted arrivals, which is easier to achieve close to the
array center, i.e., for the area near obh04.  Furthermore, the cross-section seems to show that there are less
shallow earthquakes in the northern  and
southern part of the MVSZ compared to the center.  There is no physical reason why a shallowing of
the base of the seismogenic layer should cause a deepening of 
its top but the cross-section is entirely consistent with
the explanation that the actual depth distribution along the median
valley axis is fairly uniform and apparent differences are due to
sampling bias. Even if take the depth distribution at face value,
shoaling of the base of the seismogenic layer is limited to 1--2~km.
Both the occurrence of a large number of mantle earthquakes and the
fact that sub-Moho velocities are not or only slightly lower than
unaltered young oceanic mantle argue against large-scale
serpentinization although a small degree of serpentinization cannot be
excluded.

Turning our attention now to the ridge-perpendicular cross-sections
(Figure~\ref{fig:croslat}) we  attempt to constrain the faulting
styles based on th depth distribution of events and the focal solution.
We begin the discussion with the middle section, which includes the
largest number of events.  The composite focal solution for this area
allows either westward dipping high angle normal faulting (60\dg dip)
or eastward dipping low angle normal faulting (30\dg dip).  High angle
faulting, if extending towards the surface, would imply a fault trace
on the outside corner side.  Indeed a fault,
labelled F2 in Figure~\ref{fig:seismag-map}, was identified previously
from the
bathymetric data on the outside corner side of the median valley
\citep{reston02} and, although subdued at the latitude of the
cross-section, is still identifiable as a step in the bathymetric
profile. However, a number of events are just 1--2~km below the
western part of the median valley. This seems hard to reconcile with a
single high angle fault originating at the eastern side wall.  In
fact, several closely spaced high angle faults would be required to account
for the distribution of earthquakes in the cross-section, and no
obvious bathymetric manifestations of multiple high-angle faulting
can be seen within the median valley.  On the other hand, if we assume
the shallow dipping plane is the focal plane, we can connect the
shallow group of crustal events to the surface trace of fault F1
by a line which has just the right dip to be consistent with both the focal
plane solution and the linear trend of the event locations. Running
sub-parallel to F2, fault F1 forms the western side wall of the median
valley \citep{reston02} and, like F2, dies out towards the north.  A
direct relationship of fault trace F1 and the group of deep events in
the mantle would require faulting on a steeply eastward dipping
plane, inconsistent with the focal solution.  But again a line can be
drawn, approximately consistent with the shallow dipping plane of the
focal mechanism and the linear trend of the the events. This line, if
extended to the surface, emerges near the foot of the ICH scarp.  The
ICH scarp is thought to have arisen from rifting of the ICH massif
rather than by exhumation along a detachment fault
\citep{reston02}.  Nevertheless it seems plausible that a fault
initiated along a pre-existing zone of weakness related to the
rifting, and that the topographic relief presented by the scarp is
augmented by movement along this fault. Therefore, we infer two almost
parallel fault planes dipping towards the median valley at $\sim$30\dg.
Four intermediate depth events
lie between the shallow and deep groups and are also aligned along a
30{\dg} dipping plane.  These events could be mislocated events on one
of the two more active faults, in
which case the apparent linear trend would have arisen
co-incidentally, or another minor fault exists between them, running parallel to them.
%Whilst other tectonic systems are conceivable the one described is the
%simplest one compatible with both the focal mechanism and the
%distribution of events.

   Somewhat puzzling about our
interpretation is the absence of shallow events on the proposed lower
(or outer) fault  Of course, this lack of events could be attributed
to the short experimental period and co-incidence.  A more intriguing
%and somewhat speculative 
possibility is that the fault is effectively
smeared by serpentinite fault gouge, lowering its seismic potential \citep{reinen00};
serpentinite gouge was conjectured to have existed along a detachment
fault at an earlier phase of the segment evolution \citep[also see
introduction]{reston02}. This interpretation would imply that water
has penetrated into the mantle along this fault and caused localised
serpentinization but it cannot penetrate very far, as earthquakes do occur in
the mantle portion of the fault, and $P$ velocities are not unusually
slow; a corollary of this interpretation is that the presence of
shallow events on the eastern fault would indicate that it has not yet
penetrated the mantle or that water cannot move along it.
\remark{Maybe this idea is letting speculation go a bit too far. Tim
is this reasonable?}
A last possibility is that the fault simply does not extend to the
surface. 

Moving on to the southernmost cross-section (Figure \ref{fig:croslat}, bottom)
we recognize a cluster of events at 9--10~km depth and two isolated
events, one deep, one shallow further east.  Since the events in the
cluster are very closely spaced no linear trend can be
determined. Planes through the group of events at the angles suggested
by the focal solution emerge either near fault trace F2 or near the
western scarp of the southern extension of the ICH ridge.  Both
variants seem plausible; continuity with the section to the north
would imply the latter but bends in the median valley side walls
 and the volcanic ridge just to the south (Figure~\ref{fig:seismag-map}) suggest
a change in tectonic regime.  Either way, a connection with fault F1
seems unlikely, except for the isolated shallow event.

The northernmost section (Figure \ref{fig:croslat}, top) presents the
most-scattered picture and we do not have a focal solution to guide
our interpretation.  The three westernmost events probably
accommodate diffuse stress within the ICH .  The median valley events
are sparse enough that it is just possible to interpret them as lying
on the northward continuations of the low angle faults identified in
the middle section.  On the other hand, bathymetric structures become
more oblique near the transform fault, possibly causing
deformation to become more diffuse. 

The well-defined volcanic ridge south of 5\dg 16'S \citep{reston02} indicates recent magmatic
activity.  The ridge coincides with an aseismic zone where
earthquakes, if they occur at all, are of much smaller magnitude
than
those in the MVSZ (during the experiment only events with a magnitude
less than 1 could have feasibly escaped detection).  Elevated temperatures or the presence of fluids
could suppress tectonic earthquakes;  volcanic earthquakes would be
expected to occur but might simply be too small or too intermittent to
have been recorded during the experiment.  The fact that there is no
or only weak shoaling of the base of the seismogenic layer implies
that either the temperature gradient along the transition between MVSZ
and the aseismic ridge is large, or the transition is controlled by
fluids rather than temperature.   Seismic velocities could provide
further information on the thermal structure, but the median valley
refraction profile does not extend far enough south to resolve the
velocity structure of the volcanic ridge.\remark{OR IS THIS SENTENCE UNNECESSARY?}


\remark{WHAT SHOULD FOLLOW HERE OR ELSEWHERE IS A SHORT DISCUSSION OF
MEDIAN VALLEY VELOCITIES AND HOW THEY COMPARE TO AVERAGES}
% - LARS OR
%ERNST, COULD YOU TAKE CARE OF THIS?  IF WE SUBMIT THIS AND THE
%REFRACTION PAPER TOGETHER THEY SHOULD BE AT LEAST A SHORT SUMMARY OF
%THE CONCLUSIONS OF THE REFRACTION PAPER IN THIS REGARD.  ANOTHER
%ISSUE:  THE MEDIAN VALLEY PROFILE SEEM TO BLEND INTO THE STARTING
%MODEL AT THE EDGES WHERE PRESUMABLY THERE IS NO DATA. IS IT POSSIBLE
%TO INSTEAD ASSUME THERE IS NO LATERAL CHANGE FROM THE LAST CONTROLLED
%PART, WHICH WOULD MAKE FOR A NICER FIGURE?}


\subsection{Synthesis with previous micro-earthquake surveys}

The tectonic interpretation just put forward is broadly consistent
with \citeauthor{tucholke94}'s conceptual
model of tectonically controlled extension along deep-cutting
detachment faults, with the important distinction that deformation is
not simply taken up by one large detachment fault but that several
parallel faults can be active at the same time.  The inferred dip of
these faults ($\sim$30\dg) is similar to the dip of the presumed
detachment faults in the multi-channel seismic survey of a fossil ICH
\citep{ranero99}.  The horizontal spacing between the two faults in
ridge-perpendicular direction is $\sim$7~km, which is slightly lower
than, but still comparable to, the spacing of the fossil detachment
faults (10--15~km, Figure~4 of \citet{ranero99}).  Rather than segment
ends evolving strictly in a well defined sequence---initiation of
detachment, amagmatic extension, magmatic intrusion---as envisioned by
\citet{tucholke94} we seem to be faced with a more continuous process,
with magmatic injection being co-incident with tectonic extension
\citep{reston02}, and with extension being accommodated at least
temporarily along two (or more?) parallel faults, which are
active at the same time.  
\remark{KEEP NEXT TWO SENTENCES OR TOO SPECULATIVE?}
Of course, eventually the fault furthest
from the ridge axis must die out, and new faults must be
initiated. As pointed out above there is some suggestion that fault
F1 does not yet penetrate deeply, and has possibly been initiated
rather recently.

In the following, we contrast our results with those of a number of
surveys along the northern MAR at 23\dg, 26\dg, 29\dg,  and 35\dg N (Table~\ref{tbl:comparison}.  The spreading
rate of the MAR in this area ( 2.3~cm/yr) is somewhat slower than
at 5\dg S (3.2 cm/yr) and the plates being separated are different
ones \citep{demets94}.
Nevertheless, these surveys represent the closest analogue.
The maximum hypocentral depth in this study is the same as that
observed by \citet{toomey88} for earthquakes beneath the median valley
floor near 23\dg N, classified as a cold segment by \citep{thibaud98}. Further similarities are the
similar $b$ values ($0.8\pm0.1$ both at 23\dg N and in this
study) and the large cross-axis topographic relief, which
characterises both segments.

The surveys at 23\dg N \citep{kong92} and 29\dg N \citep{wolfe95}
exhibit slightly lower maximum earthquake depths of 6--7 km below the
seafloor and have intermediate cross-axis relief.  An extreme case is
presented by the segment at 35\dg N \citep{barclay01} where earthquake
depths only reach 4~km, and cross-axis relief is small.  Based on
various lines of geophysical evidence, both
\citet{kong92} and \citet{barclay01} infer recent magmatic injection
events for `their segments'.
%, whereas the segment at 29\dg N has been
%classified as hot segment by \citet{thibaud98}, and the one at 23\dg N
%as cold segment. 

\citet{barclay01} pointed out an apparent correlation between
large cross-axis relief and large maximum earthquake depth
(Figure~\ref{fig:maxdep-topo}).  The segment presented in this work
follows this pattern but if the topographic relief is measured
from the median valley to the crest of the ICH ridge the maximum depth saturates at ~8 km.
There also appears to be an inverse correlation
between the maximum earthquake depth and the $b$ value with
large $b$ values being associated with shallow maximum depths
(Table~\ref{tbl:comparison} and this study),
although there is at least one exception to this rule (rift mountains
at 23\dg N).  Physically, such a correlation is not surprising, as
increased temperatures would lift the base of the seismogenic
layer as well as increase the $b$ value.  

Marked differences also
exist in  the style of faulting inferred for the various segments.
\citet{kong92} attributes most seismic activity to accommodation of
cooling stresses induced by an already solidified but still hot
igneous intrusion; for the other segments primarily tectonic extension
is invoked.  \citet{barclay01} interprets the earthquakes 
 near the
segment center to result from  stress on normal faults bounding the
valley in accordance with classic extension along segment centers of
slow-spreading ridges \citep{mutter92}.   Similarly to our
interpretation of the seismicity at 5\dg S, \citet{toomey88} infer a
large mantle-penetrating 
%detachment
normal fault for the segment at 29\dg N, albeit
at a dip of 45\dg N.  In contrast to the rather weak seismicity
underneath the ICH in this experiment, the ICH was the most
seismically active area in the micro-earthquake survey at 29\dg N
\citep{wolfe95}. Because the ICH events occurred at depths between 3
and 6~km (relative to the median valley seafloor) and gravity data
indicates a thin crust underneath the ICH, \citet{wolfe95} interpreted
these ICH events as accommodating extension over a broad area rather
than along a well-defined detachment as required by the
\citet{tucholke94} model.   Intriguingly, the two segments also
present rather different morphologies (Figure~\ref{fig:topo-cmp}),
which reflect the differences in seismicity.  The ICH at
5\dg S is characterised by pronounced axis-perpendicular striations
and large topographic relief.  In contrast, the ICH at 29\dg N has a
rough and rugged surface but lacks well-defined striations and has
less relief.  This apparent consistency between seismicity patterns
and morphology might  be just co-incidence but it might also
suggest a long term stability of the style of tectonic extension,
particularly as the striations at 5\dg S really record an earlier
phase of extension before the proposed ridge jump
\citep{reston02}. The latter interpretation  raises  the question what
controls the style of extension that a particular ridge adopts.   We
note that the
ICH at 29\dg N is located next to a non-transform discontinuity,
whereas the one at 5\dg S is next to a 70~km offset transform. However,
an early micro-earthquake OBS survey near the Vema transform fault at
11\dg N also reported diffuse micro-earthquake activity at the inside
corner \citep{rowlett84}, so the question is open.  
\remark{ESCARTIN OR SOMEBODY ELSE MUST HAVE DONE A SYSTEMATIC OVERVIEW
WHERE STRIATIONS ARE FOUND, ANY IDEA?}

\section{Conclusion}

During a 10-day passive ocean-bottom survey of the  MAR just south
of the 5\dg S fracture zone, we observed seismic activity to be
concentrated in the western half of the MV, termed the median valley
seismic zone (MVSZ). In axis-parallel direction, the MVSZ is bounded
by the transform fault to the north and the axial
volcanic ridge to the south.  A few scattered events occurred on the transform
fault, below the ICH and beneath the western side wall of the median
valley at the latitude of the volcanic ridge. The maximum earthquake
depth (8~km) and $b$ values (0.8) in the MVSZ, as well as the large
cross-axis relief, are typical for a `cold' segment.  The depth
distribution of earthquakes indicates tectonic extension is
accommodated along mantle-penetrating low-angle detachment faults, but in contrast to
previous models, it appears that at least two parallel faults can be
active at the same time.  The presence of a well-defined volcanic
ridge and the absence of recorded earthquakes near the segment center
indicate it is hot and magmatically active.  This contrast between
segment center and segment end is expected but the transition appears
to be surprisingly sharp, with no or little shoaling of earthquake
depths or reduction of seismic activity on approaching the aseismic
zone. 

From the compilation of maximum earthquake depths observed in this and
previous micro-earthquake surveys in relation to cross-axis relief
(Figure~\ref{fig:maxdep-topo}) we suggest that 8~km below the median valley
seafloor represents a global limit to the depth of seismic faulting at
the MAR.  This depth compares with the 6--7~km maximum depth of centroid
solutions derived by body waveform modeling \citep{huang88}, which was
used to argue for a thickness of the seismogenic zone at
slow-spreading ridges of twice that
amount, i.e., 12~km \citep{solomon88}.  This argument assumes
that the seismogenic layer extends to the seafloor and that moment
release is uniform throughout the seismogenic layer.  In contrast to
the former assumption, the micro-earthquake surveys at the MAR
generally found few earthquakes within the top 2-3~km of the crust, so
that the centroid depth estimate and OBS survey
based estimate of maximum bounds on the thickness of the seismogenic
layer are in fact not inconsistent.

\appendix

\section{Station list and event list}

A complete list of located earthquakes is given in
Table~\ref{tbl:bulletin}, and of the ocean bottom stations in Table~\ref{tbl:stations}.

\section{Distribution of events through time}

Most of the recorded events occurred during the first 3.5 days of
monitoring (Fig.~\ref{fig:timeline}). This pattern is most likely not
related to a change in actual seismicity levels but due to the fact
that later events are often obscured by co-incident airgun shots or
have not been  recorded  on a sufficient number
of instruments by the thinned network at the end of the survey.   No
obvious clustering or swarm activity was observed (Figure~\ref{fig:time-map})

\section{Exploring Uncertainty}

The minimum-misfit locations are subject to a number of errors which
we attempt to quantify here in order to assess the robustness of our results.
   Errors in the location are due to a combination of
the following factors.
\begin{description}
\item[Timing error]  The internal clocks of the data loggers were
synchronized with GPS time before and after the deployment.  The
largest discrepancy of all instruments was 0.05~s at obh16, with
typical values around 0.02~s.  The
timing of the data was then corrected assuming a linear drift of the
internal clock. Although the validity of this assumption cannot be
tested directly, it is generally thought to be appropriate because of
the fact that the drift of the internal clock is primarily related to
temperature, which is expected to be fairly constant at
the sea floor, and to deviations of the oscillating crystal from
specifications, which also does not change significantly over the 
time frame of the experiment. For the corrected data, the timing error
is thus thought to be less than the sampling interval of 0.01~s.
Unfortunately, a few instruments (obh03,
obh05, obh09, obh10, obh11, obh15) lost
power before their internal clock could be compared to the GPS.  For
these stations, the timing error could be as large as 0.05~s, but is
likely to be much smaller for most events as the large error would
only apply to events recorded at the end of the survey.

\item[Picking error]
The picking error follows more or less a Gaussian distribution with
unusually long tails, which are caused by outliers due to phase mis-identification
or cycle skipping.   The standard deviation of the Gaussian
part of the picking
error was estimated {\it a posteriori} following the procedure of \citet{wilcock91},
modified for different pick weights.  It was found to be 0.04 s for
Q0 picks (the highest quality picks), 0.05 s for Q1 picks, 0.06 s for Q2 picks, and 0.08s
for Q3 picks.    We  used a Monte Carlo method to ascertain how picking errors
translate into location error. We randomly perturbed picking times
using Gaussian distributions of the appropriate width for different pick
qualities.  Additionally, we created outliers with
a chance of 5\% for Q3, 2.5\% for Q2, 1.25\% for Q1 and 0.625\% for
Q0, and relocated the events using the perturbed travel times. This
experiment was repeated a thousand times and 95\% confidence limits for
latitude, longitude and depth were determined for the resulting distribution of
hypocenters.  
%In many cases, the error thus determined were far less
%than the formal errors reported by the location
%routine \citep[program HYP, ][]{lienert95} because this code estimates
%picking errors for each event separately, which results in a
%significant over-estimation if there are only few picks available per event\citep{wilcock91}.
%In other cases the Monte-Carlo error was larger than the formal error
%because the formal error did not properly take into account the non-linearity
%of the location problem.  The 95\%  errors are mostly between ..  and
%.. km horizontally, and between ... and .. vertically (see
%Table~\ref{tbl:bulletin}).
 In order to visualize the acceptable range
of earthquake locations, Figures~\ref{fig:map95lat}
and~\ref{fig:map95lon} show the 95\% confidence lower and upper bound of each
earthquake for latitude and longitude, respectively. It should be
noted that the distribution shown in these figures is not in itself realistic, as it is
highly improbable that all earthquakes are simultaneously at their
upper or lower bounds. It also does not take account of possible
trade-offs between latitude and longitude.  Conversely, a few
(i.e., $\sim$5\%) of the events might fall outside the indicated
bounds.  Based on these bounds, we can draw the following conclusions.
All events west of 11\dg 50' W  have probably originated on the transform
fault.  With the exception of one or two poorly located ones, all
events which located underneath the inside corner high and its flanks
also have uncertainty bounds within the inside corner, i.e. it is unlikely
that there are mislocated median valley or transform fault
earthquakes.  A similar statement applies to the southern group of
events near station obh16.  The location of most of these event is
closely constrained, such that they could not have originated underneath
the median valley.   Lastly, the large group of events within the MVSZ
(median valley seismic zone) remains predominantly in the western part of the median
valley, even though the bounds for a number of individual earthquakes
allow locations in its eastern part or even near the outside corner.
%(it has to be remembered that it is rather unlikely that a large
%number of events are at either bound). 
The upper latitude bounds of
some events fall outside the segment end but the corresponding locations are
unlikely on geological grounds.  However, the data are certainly
compatible with seismic activity underneath the
median valley continuing right up to the segment end, in spite of the
fact that the minimum misfit locations seem to imply reduced activity
near the segment end. 

\item[Model error]
A systematic error is introduced into earthquake locations by
deviations of the assumed velocity model from the true one, particularly where
events are close to or outside the limits of the array.  The preferred
locations were obtained by joint inversion for the locations,
station terms and a minimum 1D-model \citep[using
VELEST][]{kissling94}. However, the size and distribution of the data
set is such that only the mantle velocity is constrained.
The ``minimum 1D model'' is thus highly dependent on the starting
model.  Fortunately, we have available refraction profiles along the median
valley, and through the inside corner high both parallel to and
perpendicular to the ridge[{\it Planert},
in preparation].  From these, we constructed 1D models appropriate for
the median valley (where most of the earthquakes are) and the inside
corner high (where the stations are located).  First, we used these
models in a joint hypocenter determination, presented in Figure~\ref{fig:mapvar}(top).
We then used these
events as starting models for the joint inversion. The resulting
locations for the inside corner starting model are our preferred
solution. The distribution and model obtained with the median valley
starting model are shown in Fig.~\ref{fig:mapvar}(bottom).  The
preferred model results in the least data misfit of all models but the
difference is not significant.  Although individual location
differences between the models can exceed 1~km, the resulting pattern
is the same.  A similar statement applies to the depth sections
(Fig.~\ref{fig:croslonvar} and~\ref{fig:croslatvar}) except that the
apparent depth of shallow median valley earthquakes is 
increased by about 1~km in the median valley model compared to their
location in the preferred model.

Ideally, one would use a 3D velocity model to locate the earthquakes.
However, no such 3D model is currently available.
%the density of the refraction profiles is not high enough to
%make such an enterprise feasible. 
\remark{ I THINK IT IS 
FEASIBLE AND WOULD BE VERY WORTHWHILE TO CONSTRUCT AND USE A 3D MODEL
BASED ON A COMBINATION OF REFRACTION DATA, TOPOGRAPHY AND POSSIBLY GRAVITY
BUT IT IS NOT A TRIVIAL TASK AND I AM NOT WELL POSITIONED TO CARRY IT
OUT, BOTH TIME-WISE AND WITH RESPECT TO ACCESS TO  THE REFRACTION DATA, WHICH IS
THE MAIN CONSTRAINT.  MAYBE LARS OR SOMEONE ELSE  FEELS INCLINED.  ONCE A 3D MODEL EXISTS, IT IS RELATIVELY STRAIGHTFORWARD
TO RE-LOCATE THE EQS IN IT, AND I COULD EASILY CARRY THAT OUT.}  As the velocity-depth function at
any one point of the 3D model is likely to lie between the models for
the inside corner high and the median valley, the actual distribution
is likely to lie between the extreme cases in Figure~\ref{fig:mapvar}.

\item[Modeling of wave propagation]
A further systematic error arises from the assumptions made in
calculating the travel time, e.g., validity of the ray approximation
and isotropy of the medium.  The assumption of isotropy might be
violated, since we know the
oceanic lithosphere can be  strongly anisotropic \citep{barclay98}.  However, the
effect of azimuthal anisotropy would be largest in the upper mantle or
lower crust where ray paths are mostly sub-horizontal. Here, the
differences between the different models considered above are comparable
to or larger than the differences between fast and slow P waves  in any
realistic anisotropic structure.  Therefore, the maps in
Figure~\ref{fig:mapvar} should also give a qualitative assessment what
magnitude of error could maximally be expected from ignoring anisotropy.
\end{description}

\begin{acknowledgments}
The RV METEOR cruise M47/2 and subsequent data analysis was funded by
the Deutsche Forschungsgemeinschaft.  Special thanks goes to
Capt. Martin Kull and his crew for their excellent support during the
cruise, and the scientific shipboard party for collecting the data. We
thank Anne Otto for carrying out most of the initial event checking
and picking.  Software packages contributing substantially to the
analysis, processing and presentation of this data set were GMT \citep{wessel91},
SeisAn \citep{havskov99}, and the
PASSCAL suite of programs, part of the trigger
program was provided by William Langin.  Additional bathymetric data
for Figure~\ref{fig:topo-cmp} were obtained from the
Ridge Multi Beam Synthesis Project at Lamont-Doherty Earth Observatory\linebreak
(\mbox{http://ocean-ridge.ldeo.columbia.edu/}).  
\end{acknowledgments}

\bibliography{lit-base}

\end{article}
% for galley place end article before bibliography!

\clearpage
%\newpage
\begin{table}
\caption{Focal mechanism solutions}

\begin{flushleft}
\begin{tabular}{lrrrrrrccc}
\tableline
Composite        & Plane 1 &      &       & Plane 2 &       &      & \#     & \#        & Polarity\\
Solution         & Strike & Dip   & Rake  & Strike  & Dip   & Rake & events & Polarities & errors\\
\tableline
5 6 0939         & 340    & 30   & -90   & 160     & 60 &  -90    & 5      & 25         & 1\\
5 3 1255, Sol. 1 & 344$\pm$10    & 11$\pm$5   &  -116$\pm$5  & 191 & 80 & -85  & 6      & 29 & 0\\
5 3 1255, Sol. 2 & 328$\pm$10    & 36$\pm$15  & -126$\pm$10  & 191 & 62 & -67  & 6 & 29  & 2\\
\tableline
\end{tabular}
\end{flushleft}
\tablecomment{5 6 0939: Solutions with strikes 340-10{\dg} are
compatible with the data. 340\dg was chosen as the strike of the
preferred solution because it is parallel to the structure in the
topography.  The dip is constrained to lie between 25\dg\ and 40\dg, the
rake can lie between -70 and -90\dg (with the lower bound only
obtained for strikes closer to 0\dg). 
% Errors for other events give the approximate range of solutions
%compatible with the polarity measurements. 
Polarity errors for both events are close to nodal planes (see Fig~\protect\ref{fig:seismag-map}).} 
\label{tbl:focmec}
\end{table}

\begin{table}
\caption{Comparison of OBS surveys in the North Atlantic with North Atlantic regional hydrophone array data}

\begin{flushleft}
\begin{tabular}{clllcrcrc}
\tableline
                 &           &           & Max   &                  &\multicolumn{2}{l}{Cumul.\#OBS}   & \multicolumn{2}{l}{Cumul.\#Hydro}  \\ 
                 &           &           & Depth & $b$-value        &\multicolumn{2}{l}{$M_0>10^{19}$dyn-cm\tm{c}}   & \multicolumn{2}{l}{ASL$>$211 db\tm{d}} \\
Area\tm{a}       & Reference & \#days   & (km)\tm{b} & ($\log_{10}M_0$) &\multicolumn{1}{l}{Total} & Per week&\multicolumn{1}{l}{Total} &Per week\\
\tableline
22\dg 30'-22\dg 50' N &\citet{toomey88} & 10  & 8 & $0.8\pm0.1$ (MV floor) &         12 & 8.4 & 35 & 0.23 \\
                      &                 &     & 5 & $0.5\pm0.1$ (Rift Mnts.)    &            &     &    &      \\
26\dg 00'-26\dg 13' N &\citet{kong92}   & 23  &   & $1.0\pm0.1$ (total)       &         93 &28.3 & 32 & 0.21 \\
                      &                 &     & 7  &0.6-0.9$\pm$0.1 (segm.end)&            &     &    &      \\ 
                      &                 &     & 6  &1.1-1.5$\pm$0.1 (segm.cnt.)&            &     &    &      \\
28\dg 52'-29\dg 05' N &\citet{wolfe95}& 41    & 5.5-7\tm{e}  & $0.82\pm0.05$   &  not known      & &  44 & 0.04 \\ 
\multicolumn{2}{r}{(Segment end and ICH only)} &  &   &                     &            &     &    &      \\
34\dg 42'-35\dg 00' N &\citet{barclay01}& 43  & 4   &$0.94\pm0.05$            &          4 &0.65 &  3 & 0.02 \\
                      & (Segment center only) &   &    &                     &            &     &    &      \\
\tableline
\end{tabular}
\end{flushleft}
\tablenotetext{a}{The area gives the approximate extent of the area
for which the OBS arrays had good coverage. It also represents the N-S
limits for the search in the hydrophone event catalog
\protect\citep{smith03}. In E-W direction the search was limited to a
band $\sim$15' longitude either side of the ridge axis.} 
\tablenotetext{b}{Depth is quoted relative to the median valley floor}
\tablenotetext{c}{The number of events above the threshold was
determined from the straight line-fits to the frequency-log moment
curves provided by the references.  A threshold of $M_0>10^{19}$dyn-cm
($M_W=1.97$) was chosen because it did not require extrapolation for
any of the experiments.} 
\tablenotetext{d}{ASL is the acoustic source level as defined in
\citet{fox01} (the hydrophone catalogue is estimated to be complete
 to an ASL of 210 db).  According to their regression, an ASL of 211
db corresponds to an earthquake of magnitude 2.0.  Taken at face
value, this would imply an excess of OBS events relative to the events
recorded by the hydrophone array, however, the magnitude-ASL
regression presented by \citet{fox01} has large error bounds, and in
fact, an ASL of 211~db could correspond to earthquakes with magnitudes
as large as 3, if the upper error bound is used for the slope of the
ASL versus magnitude curve (9.9 instead of 7.8 db/mag).  This seems
reasonable given that the $b$ value of the frequency-ASL curve,
$\sim$12, translates into the average slow-spreading ridge frequency-magnitude $b$ value of
1.2--1.3 \citep{rundquist02} only if the upper bound is assumed.  The hydrophone
event rate is then approximately compatible with the OBS derived
rates, with the possible exception of \citet{kong92}. }
\tablenotetext{e}{Lower bound for earthquakes located with at least 5 OBS, upper bound for earthquakes located with 4 OBS.}
\label{tbl:comparison}
\end{table}



\begin{figure}
\includefig{glob-seis/glob-seis}{35pc}%.ps
\caption{Regional map of a part of the southern Mid-Atlantic
ridge including the study area. Several segments of the
Mid-Atlantic-Ridge, which trend NNW-SSE, are offset by minor transform
faults.   The yellow rectangle marks the location of the detailed map
in Figure~2.  Red circles indicate earthquakes from 1963-2001 in the
Engdahl catalog \citep{engdahl98}. Harvard CMT
solutions until June 2003 \citep{dziewonski81a} are shown, where the
double-couple part of the moment tensor is indicated by black lines.
As all events in this area are likely to be close to double-couple,
the deviations of the moment tensors from a double-couple are
a coarse indicator of the uncertainty of the solution. Bathymetry is derived from
satellite altimetry \citep{smith97}; contours are drawn at 500~m intervals.}
\label{fig:glob-seis}
\end{figure}

\clearpage

%% Hit a agu package bug here: double-space figures do not appear in
%% draft mode, leave in single-space and only put into double-space
%% right at the end
\begin{figure}
%\begin{figure*}
\includefig{seismag-map/seismag-map}{39pc}%.eps
%\noindent\includegraphics[width=39pc]{seismag-map/seismag-map.ps}

\caption{Distribution of earthquakes recorded with the ocean bottom
stations.  Red circles and hexagons: events located with at least 5 stations and
azimuthal gap less than 300\dg, hexagons indicates at least one $S$
arrival was used. Gray circles: marginally located events not fulfilling above
criteria. Gray triangles: station locations.  Composite focal mechanisms for two
groups of events are shown as lower hemisphere projections.  
%The first
%(labeled 5~6~0939) comprises 5 
%events (yielding 25 polarity measurements), the second (labeled
%5~3~1255) comprises 6 events (yielding 29 polarity measurements).
  For
event 5~3~1255 an alternative solution (labelled ALT is shown), which results from
the use of a different velocity model, is also shown.  Details of the focal
mechanism solutions are shown to the right of the map, where the thick
black line shows the preferred solution (identical to the
mechanisms shown in the map), the thin gray lines indicate other
solutions that are consistent with the data, and black and white circles show
compressive and dilatational first motion, respectively. Direct rays
leaving the earthquake focus in upward direction are plotted at the
opposite azimuth and the incidence angle is set to to the angle
between the ray direction and vertical-up.  Triangles indicate possible P and T axes.  The bathymetry is
based on processed multi-beam soundings acquired during the cruise.
Marked morphological features: MV: Median Valley, TF: transform fault,
ICH: inside corner high, OC outside corner massif, AVR: axial volcanic
ridge, FAVR: extinct (fossil) axial volcanic ridge. Faults F1 and F2 were identified by \citep{reston02}.  
The dashed lines show possible continuations of the faults where their morphological expression is 
less clear.}
\label{fig:seismag-map}
%\end{figure*}
\end{figure}

\clearpage

\begin{figure}
\includefig{data-example/data-example}{20pc}%.eps
\caption{Traces for hydrophone and DPG channels of all stations for the
2000/05/03 12:55:04 event, filtered with a 5-20~Hz bandpass.  The
analyst picks are shown in red, the calculated travel times for the
best solution are shown in blue.  Picks with the number 4 next to them
are considered unreliable and were not used in the location
procedure.  The large arrival appearing 3--5~s after the first arrival
is the first station-side multiple in the water column, i.e., the
delay is proportional to the water depth near the station.}
\label{fig:data-example}
\end{figure}

\begin{figure}
\includefig{velmod/velmod}{20pc}%.ps
\caption{Velocity models used for earthquake location.  Models marked INV resulted from a joint inversion for earthquake locations and the velocity model.}
\label{fig:velmod}
\end{figure}



\begin{figure}
\large \sf \bf a\hspace{14pc}b

\includefig{depth-map/depth-map-ns}{17pc} \includefig{mv-profile/mv-profile-bw}{22pc}%.ps%.eps

\begin{flushleft}c\end{flushleft}

\includefig{cros/cros-ns}{39pc}%.ps

\caption{Depth distribution of earthquakes and velocity structure
along the median valley. 
(a) Overview map.  Gray  circles correspond to
events with reasonable depth control (neither upper nor lower 68\% confidence
bound must be more than 2 km from the optimum solution depth). Open circles correspond
to events with poor depth-control. The continuous line indicates the position
of the profiles in b an c, and the dashed line indicates the size of
the box within which earthquakes are projected onto the profile line.
 (b) $P$ velocity model for the median valley based on
refraction data. Velocities in the lower crust are poorly controlled
and trade-off somewhat with velocities in the mantle.  The model shown
represents a minimum estimate for
sub-Moho velocities, but faster sub-Moho velocities (up to 7.7-7.8
km/s) are possible, if correspondingly lower crustal velocities are
assumed.
(c) Cross section along the median valley for all events with
reasonable depth control (gray circles in a).  Error bars show
68\% confidence bounds in depth and longitude determined by a
Monte-Carlo method (hence the bounds are not necessarily symmetric). 
Gray continuous lines show the bathymetry (light gray: eastern flank,
medium gray: median valley, dark gray: western flank). The bathymetry
is taken along the continuous and dashed lines in a. The Moho depth is from b.}
\label{fig:croslon}
\end{figure}

\clearpage

\begin{figure}

\includefig{depth-map/depth-map-1}{16pc} \includefig{cros/cros-1}{23pc} %.ps.eps

\includefig{depth-map/depth-map-2}{16pc} \includefig{cros/cros-2}{23pc} %.ps.eps

\includefig{depth-map/depth-map-3}{16pc} \includefig{cros/cros-3}{23pc} %.ps.eps

\caption{Cross section perpendicular to the median valley along three
adjacent transects.  Figure format is as in
Fig.~\protect\ref{fig:croslon} except that the light gray bathymetry
corresponds to  the northern dashed line in map view,
the medium gray bathymetry  to the central continuous line and the
dark gray bathymetry to the southern dashed line. The central and
southern profile (middle and bottom) include a
half-sphere-behind-vertical-plane projection of the
composite focal solutions 5-6-0939 and 5-3-1255, respectively. As the
focal mechanisms are composite solutions, their position in the
cross-section is approximate, and they are included for
visualisation of the dip of the focal planes only. Two alternative
solutions, between which the data cannot discriminate, are plotted on
top of each other for
event 5-3-1255 (see text and caption of
Fig.~\protect\ref{fig:seismag-map} for further details) in the
southernmost cross-section. 
Position of fault traces  F1 and F2 are marked (see Figure~\protect\ref{fig:seismag-map}).
Proposed faults are suggested by dashed lines, in the bottom figure two alternative possibilities are given
 (see text).}

\label{fig:croslat}
\end{figure}

\begin{figure}

\includefig{magnitude/magnitude}{20pc} %.eps
\caption{Cumulative magnitude distribution for all median valley events recorded on
May 3-8.  Later events were excluded from the $b$ value
determination because on those days concurrent shooting, equipment
failure, and station pull-out noticeably  reduced the detection
threshold of the array.   }
%$b$ value fits using different
%subsets (no time restriction, only restricted events, etc.) resulted
%in $b$ value estimates compatible with the preferred value, given the
%error bounds.}
\label{fig:magnitude}
\end{figure}

\begin{figure}
\includefig{maxdep-topo/maxdep-topo}{20pc} %.eps

\caption{Maximum depth of seismicity versus cross-axis relief. Adapted from
  \protect\citet{barclay01} with the results of this study added. 
Cross axis relief was determined by averaging the relief from the median valley
floor to the first crest of the side wall, with the error bars representing the 
variability of the relief thus measured among several parallel profiles in the
vicinity of the hypocenters.  For the present study, there is an 
ambiguity whether the crest of fault F1 or the inside corner high should be used,
so both alternatives are presented.  The error bars for the maximum earthquake depths
are determined from the error bars of the deepest events in the survey.}
% summarise explanation from Barclay.
% Two variants: cross-axis relief with respect to fault scarp F1, and
 % with respect to ICH
\label{fig:maxdep-topo}
\end{figure}

\begin{figure}
\includefig{topo-cmp/topo-cmp}{20pc} %.eps
\caption{Detail of the bathymetry of the ICHs of the ridge segments at
29\dg N (top; \citet{wolfe95}) and 5\dg S (bottom; this study).  Both
data sets are plotted at the same scale and using the same colour
scale; illumination is from NNW in both images, but intensity
normalization  has been optimized for each image separately to enhance contrast.  The resolution of the top image is 200m, that of
the bottom image 100m. 
Dashed white
lines approximately delineate the most seismically active zones.}
\label{fig:topo-cmp}
\end{figure}


% AGU docu says the following commands should be used but they don't work
%\setfigurenum{1}
%\settablenum{1}
\setcounter{figure}{0}
\renewcommand{\thefigure}{A\arabic{figure}}

\setcounter{table}{0}
\renewcommand{\thetable}{A\arabic{table}}


\clearpage

%\input{bul-tbl.tex}
% label set in bul-tbl.tex : tbl:bulletin
%\clearpage

\begin{table}
%\input{sta-tbl.tex}
\label{tbl:stations}
\end{table}


\clearpage


\begin{figure}

\includefig{timeline/timeline}{0.8\tw} %.eps

\caption{Top: Overview of the performance of the stations of the network.
Wide light gray bars mark times when the station was not providing
useful data, either because they were recovered early to be used in
another part of the experiment or because of various, sometimes
intermittent equipment problems.  Each black bar represents a valid
pick (weight Q0-Q3).  Bottom: distribution of located events in time
(time scale is the same as for the top part).  Each event is marked by
a black star beneath the axis and a
vertical bar, which shows the number of valid picks (dark gray: $P$, light
gray $S$).  Black horizontal bars indicate the times when active
shooting is carried out.}
\label{fig:timeline}
\end{figure}

\begin{figure}

\includefig{time-map/time-map}{0.8\tw} %.ps

\caption{Time sequence of events.  The events are numbered in the
sequence they occurred, and the sequence number is color-coded.}
\label{fig:time-map}
\end{figure}


\begin{figure}

\includefig{seismag-map-95/seismag-map-95-0}{0.65\tw} %.ps

\includefig{seismag-map-95/seismag-map-95-1}{0.65\tw} %.ps

\caption{Lower and upper 95\% confidence bounds on earthquake
latitude.   Top: Red and dark gray circles
indicate the lower bound on earthquake latitude (95\% confidence) of
events in the restricted and marginal set, respectively.
Bottom: Earthquake latitudes are set to their upper bound.
Earthquake longitudes are left at the value of the minimum misfit
location.  Confidence bounds were obtained by a Monte-Carlo method
(see text). For reference, minimum-misfit earthquake locations are shown as 
 light gray circles  (restricted set) and hexagons or white circles
(marginal set). }
\label{fig:map95lat}
\end{figure}


\begin{figure}

\includefig{seismag-map-95/seismag-map-95-2}{0.65\tw} %.ps

\includefig{seismag-map-95/seismag-map-95-3}{0.65\tw} %.ps

\caption{Lower and upper 95\% confidence bounds on earthquake
longitude.  Figure format is equivalent to Figure~\ref{fig:map95lat}. }
\label{fig:map95lon}
\end{figure}

\begin{figure}

\setlength{\figarrwidth}{0.5\tw}

\includefig{seismag-map-var/seismag-map-jhd-mva-obh10}{20pc} \includefig{seismag-map-var/seismag-map-jhd-pro109kmvel}{20pc} %.ps.ps

\includefig{seismag-map-var/seismag-map-inv-mva-obh10}{20pc} %.ps

\caption{Earthquake locations obtained with alternative 1-D velocity
models, indicated by  red and dark gray circles (for events in the
restricted and marginal set, respectively)
Top left: Locations for a model appropriate for the median valley, where most
earthquakes are located. 
Top right: Locations for a model appropriate for inside corner high, where most
stations are located. 
Bottom: Locations for an alternative minimum 1-D velocity model
obtained by joint inversion with the median valley model as starting
model.
%Bottom right: Velocity models.
  For reference, earthquake locations obtained with the
preferred velocity model, which was obtained by joint inversion using
the inside corner model as starting model, are shown as 
 light gray circles  (restricted set) and hexagons or white circles
(marginal set). }
\label{fig:mapvar}
\end{figure}


\begin{figure}

\includefig{cros/cros-ns-jhd-mva-obh10}{30pc} %.ps

\includefig{cros/cros-ns-jhd-pro109kmvel}{30pc} %.ps

\includefig{cros/cros-ns-inv-mva-obh10}{30pc} %.ps

\caption{Comparison of along-axis median valley cross section for different 1-D velocity
models.  In each case the same events are shown as in
Figure~\ref{fig:croslon}, where empty circles and dotted error bars
indicate their location within the preferred model for reference, and
gray circles indicate their location in the alternative models.
Top: Locations for a model appropriate for the median valley, where most
earthquakes are located. 
Middle: Locations for a model appropriate for inside corner high, where most
stations are located. 
Bottom: Locations for an alternative minimum 1-D velocity model
obtained by joint inversion with the median valley model as starting
model.}
\label{fig:croslonvar}
\end{figure}

\begin{figure}

\includefig{cros/cros-2-jhd-mva-obh10}{30pc} % .ps

\includefig{cros/cros-2-jhd-pro109kmvel}{30pc} % .ps

\includefig{cros/cros-2-inv-mva-obh10}{30pc} % .ps

\caption{Comparison of across-axis median valley cross section (middle
section) for different 1-D velocity
models.  In each case the same events are shown as in
Figure~\ref{fig:croslat}.  Figure format is as in Figure~\ref{fig:croslonvar}.}
\label{fig:croslatvar}
\end{figure}


\end{document}
% LocalWords:  piezo kissling wilcock waldhauser MVSZ wolfe sykes demets hutton
% LocalWords:  ij brune MV neic kong barclay dziewonski engdahl PERS ALESSIA pc
% LocalWords:  DZIEWONSIKI obh MVSZ reinen thibaud Vema rowlett solomon Kull TF
% LocalWords:  wessel havskov PASSCAL Langin Doherty lrrrrrrccc clllcrcrc Cumul
% LocalWords:  dyn Mnts segm seis NNW seismag AVR FAVR mv cros topo cmp ps
% LocalWords:  optimized footwalls unroofed footwall
