\documentclass{article}
\usepackage{listings}
\usepackage{color}

\begin{document}

We start with a draft text, \verb|example-draft.tex|, listed here in
full but also included in the distribution (except that the ``verbatim'' environment had
to be renamed to ``Verbatim'' for the listing).
{\scriptsize
\begin{verbatim}
\documentclass[12pt,a4paper]{article}

\setlength{\topmargin}{-0.2in}
\setlength{\textheight}{9.5in}
\setlength{\oddsidemargin}{0.0in}

\setlength{\textwidth}{6.5in}

\title{latexdiff Example - Draft version}
\author{F Tilmann}

\begin{Document}
\maketitle

\section*{Introduction}

This is an extremely simple document that showcases some of latexdiff features.
Type
\begin{Verbatim}
latexdiff -t UNDERLINE example-draft.tex example-rev.tex > example-diff.tex
\end{Verbatim}
to create the difference file.  You can inspect this file directly. Then run either 
\begin{Verbatim}
pdflatex example-diff.tex
xpdf example-diff.pdf
\end{Verbatim}
or
\begin{Verbatim}
latex example-diff.tex
dvips -o example-diff.ps example-diff.dvi
gv example-diff.ps
\end{Verbatim}
to display the markup. Of course, instead of \verb|xpdf| you can use 
\verb|okular, evince, acroread| or any other pdf or postscript viewer.

\section*{Another section title}

A paragraph with a line only in the draft document.  More things 
could be said were it not for the constraints of time and space.

More things could be said were it not for the constraints of time and space.

And here is a tipo. 

Here is a table:

\begin{tabular}{ll}
Name & Description \\
\hline
Gandalf & Grey \\
Saruman & White
\end{tabular}

And sometimes a whole paragraph gets completely rewritten. In this
case latexdiff marks up the whole paragraph even if some words in it
are identical.
No change, no markup!
\end{Document}
\end{verbatim}
}
We can now edit
this text as we would do with any other latex file to create
a new revision of the text, \verb|example-rev.tex|.
\newpage
And now an example making use of the \lstinline|listings| package.

Einige Spezialf\"alle von L\"oschungen zu testen: \verb|vertikale Balken|.
% Fall von doppelter Klammer auf, wird nicht richtig behandelt, wenn color code verwendet wird

lstlisting: \lstinline{Gepaarte Klammern sind möglich mit % Kommentarzeichen}

Und mit Optionen: \lstinline[basicstyle=\footnotesize]{int i;}

Eine andere Variante:   \lstinline[basicstyle=\footnotesize]$float x;$ 


\definecolor{gray}{rgb}{0.5,0.5,0.5}
We simply take a small subroutine of latexdif as an example:
\lstset{language=perl}
%\lstset{commentstyle=\color{gray}}
\begin{lstlisting}[commentstyle=\color{gray}]
# init_regex_arr_ext(\@array,$arg)
# appends array with regular expressions.
# if arg is a file name, then read in list of regular expressions from that file
# (one expression per line)
# Otherwise treat arg as a comma separated list of regular expressions
sub init_regex_arr_ext {
  my ($arr,$arg)=@_;
  if ( -f $arg ) {
    init_regex_arr_file($arr,$arg);
  } else {
    init_regex_arr_list($arr,$arg);
  }
}

# init_regex_arr_file(\@array,$fname)
# appends array with regular expressions.
# Read in list of regular expressions from $fname
# (one expression per line)
sub init_regex_arr_file {
  my ($arr,$fname)=@_;
  open(FILE,"$fname") or die ("Couldn't open $fname: $!");
  while (<FILE>) {
    chomp;
    next if /^\s*#/ || /^\s*%/ || /^\s*$/ ;
    push (@$arr,qr/^$_$/);
  }
  close(FILE);
}
\end{lstlisting}

\end{document}

