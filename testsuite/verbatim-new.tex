\documentclass{article}
\usepackage{listings}
\usepackage{color}

\begin{document}

We edit to obain revised text, \verb|example-rev.tex|, listed here in
full but also included in the distribution (except that the ``verbatim'' environment had
to be renamed to ``Verbatim'' for the listing).
{\scriptsize
\begin{verbatim}
\documentclass[12pt,a4paper]{article}

\setlength{\topmargin}{-0.2in}
\setlength{\textheight}{9.5in}
\setlength{\oddsidemargin}{0.0in}

\setlength{\textwidth}{6in}

\title{latexdiff Example - Revised version}
\author{F Tilmann}
% Note how in the preamble visual markup is never used (even
% if some preamble might eventually end up as visible text.)

\begin{Document}
\maketitle

\section*{Introduction}

This is an extremely simple document that showcases some of the latexdiff features.
Type
\begin{Verbatim}
latexdiff -t UNDERLINE example-draft.tex example-rev.tex > example-diff.tex
\end{Verbatim}
to create the difference file.  You can inspect this file directly. Then run either 
\begin{Verbatim}
pdflatex example-diff.tex
xpdf example-diff.pdf
\end{Verbatim}
or
\begin{Verbatim}
latex example-diff.tex
dvips -o example-diff.ps example-diff.dvi
gv example-diff.ps
\end{Verbatim}
to display the markup.

\section*{Yet another section title}

 More things could be said were it not for the constraints of time and space.

A paragraph with a line only in the revised document.  
More things could be said were it not for the constraints of time and space.

And here is a typo. 

Here is a table:

\begin{tabular}{ll}
Name & Description \\
\hline
Gandalf & White \\
Saruman & Evil
\end{tabular}

And now for something completely different, with not a paragraph in sight.
No change, 
no markup!
\end{Document}
\end{verbatim}
}
We can now compare
this text to the draft version.
\newpage
A now an example making use of the \lstinline|listings| package.

Some new verbatim listings to test odd variants: \verb|vertical bars|.
% Note that \verb with two opening braces no longer works as it interferes with colored markup
In lstlisting we can actually use paired braces \lstinline{Listing brackets by braces % with comment character}.
And even options with lstlisting are possible: \lstinline[basicstyle=\footnotesize]{var i:integer;} 
Another variant of this: \lstinline[basicstyle=\footnotesize]$var x:float;$ 

\definecolor{gray}{rgb}{0.5,0.5,0.5}
We simply take a small subroutine of latexdif as an example:
\lstset{language=perl}
%\lstset{commentstyle=\color{gray}}
\begin{lstlisting}[commentstyle=\color{gray}]
# init_regex_arr_ext(\@array,$arg)
# appends array with regular expressions.
# if arg is a file name, then read in list of regular expressions from that file
# (one expression per line)
# Otherwise treat arg as a comma separated list of regular expressions
sub init_regex_arr_ext {
  my ($arr,$arg)=@_;
  init_regex_arr_list($arr,$arg);
}

# init_regex_arr_file(\@array,$fname)
# appends array with regular expressions.
# Read in list of regular expressions from $fname
# (one expression per line)
sub init_regex_arr_file {
  my ($arr,$fname)=@_;
  open(FILE,"$fname") or die ("Couldn't open $fname: $!");
  while (<FILE>) {
    chomp;
    next if /^\s*#/ || /^\s*%/ || /^\s*$/ ;
    push (@$arr,qr/^$_$/);
  }
  close(FILE);
}
\end{lstlisting}



\end{document}

